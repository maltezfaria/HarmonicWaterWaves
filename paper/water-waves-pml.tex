\documentclass[11pt]{article}

% Packages and macros go here
\usepackage{graphicx}
\usepackage{tabu}
\usepackage{subcaption}
\usepackage{lipsum}
\usepackage{amsfonts}
\usepackage{epstopdf}
\usepackage[lined,boxed]{algorithm2e}
\usepackage{booktabs}
\usepackage{subcaption}
\usepackage{esvect}
\usepackage{multirow}
\usepackage{amsmath}
\usepackage{amssymb}
\usepackage{caption}

\usepackage{changes}
\usepackage{todonotes}% See
% http://tug.ctan.org/macros/latex/contrib/changes/changes.english.pdf for documentation
\definechangesauthor[name={Luiz Faria},color=blue]{lmf}
% \newcommand{\missing}[1]{\todo[color=red]{#1}}
\newcommand{\improvement}[1]{\todo[color=orange]{#1}}
\newcommand{\question}[1]{\todo[color=yellow]{#1}}

\usepackage{multirow}
\usepackage{amsmath}
\usepackage{amsfonts}
\usepackage{amsmath}
\usepackage{amssymb}
\usepackage{graphicx}
\usepackage{caption}
\usepackage{enumitem}
\usepackage{mathrsfs}
\usepackage{upgreek}
\usepackage{amsthm}
\usepackage{booktabs}
\usepackage{authblk}
\usepackage[noabbrev]{cleveref}
\crefformat{equation}{(#2#1#3)}

\newcommand{\pder}[2]{\frac{\partial #1}{\partial #2}}
\newcommand{\R}{\mathbb{R}}
\newcommand{\C}{\mathbb{C}}
\newcommand{\N}{\mathbb{N}}
\newcommand{\bn}{\mathbf{n}}
\newcommand{\bx}{\mathbf{x}}
\newcommand{\bz}{\mathbf{z}}
\newcommand{\btau}{\mathbf{\tau}}
\newcommand{\bhx}{{\mathbf{\hat{x}}}}
\newcommand{\bhy}{\mathbf{\hat{y}}}
\newcommand{\hx}{{\hat{x}}}
\newcommand{\hy}{{\hat{y}}}
\newcommand{\hvarphi}{{\hat{\varphi}}}
\newcommand{\hPhi}{{\hat{\Phi}}}
\newcommand{\by}{\mathbf{y}}
\newcommand{\br}{\boldsymbol{r}}
\newcommand{\btheta}{\mathbf{\theta}}
\newcommand{\de}{\,\mathrm{d}}
\newcommand{\htau}{{\hat{\tau}}}
\newcommand{\hpsi}{{\hat{\psi}}}
\newcommand{\M}{\mathcal{T}}
\newcommand{\surface}{\Gamma_{f}}
\newcommand{\solid}{\Gamma_{s}}
\newcommand{\tvarphi}{\tilde \varphi}

%%%%%%%%%%%%%%%%%%%%%%
\newtheorem{theorem}{Theorem}[section]
\newtheorem{lemma}[theorem]{Lemma}
\newtheorem{proposition}[theorem]{Proposition}
\newtheorem{corollary}[theorem]{Corollary}
\newtheorem{remark}[theorem]{Remark}
\newtheorem{definition}[theorem]{Definition}

\topmargin -.5in
\oddsidemargin 0pt
\textheight 8.8in
\textwidth 6.5in

\title{Complex-scaled boundary integral equation for time-harmonic water waves}

\author[1]{Anne-Sophie Bonnet-Bendhia \thanks{anne-sophie.bonnet-bendhia@ensta-paristech.fr}}
\author[1]{Luiz M. Faria\thanks{luiz.maltez-faria@inria.fr}}
\author[2]{Carlos Perez-Arancibia\thanks{c.a.perezarancibia@utwente.nl}}
\affil[1]{\small{Laboratoire POEMS, CNRS/ENSTA/INRIA, France}}
\affil[2]{\small{Univeristy of Twente, Netherlands}}

\date{\today}

\begin{document}

\maketitle

% REQUIRED
\begin{abstract}  
  We present a novel boundary integral equation (BIE) formulation for
  time-harmonic water-waves problem based on the complexification of Laplace's
  free-space Green function. The method employs a perfectly matched layer (PML)
  coordinate-stretching to render the propagative waves exponentially decaying,
  thus facilitating the truncation of the unbounded interfaces comprised by the
  free-surface and the bottom topography. The formulation uses only simple
  function evaluations (e.g. complex logarithms and square roots), avoiding the
  use of the water-wave Green’s function. We show through a variety of numerical
  examples that the truncation errors are exponentially small with respect to a
  parameter $\ell$ controlling the length of the PML layer.
\end{abstract}

% REQUIRED
 \textbf{Keywords}: Perfectly matched layers, boundary integral equations \\

%\noindent \textbf{AMS subject classifications}:

\tableofcontents

\section{Introduction}

Solving partial differential equations (PDEs) on unbounded domains poses
additional challenges from both a theoretical and numerical point of view.
Theoretically, one is required to impose appropriate conditions at infinity so
as to guarantee the existence of a unique solution; when the underlying PDE
comes from the modeling of the physical world, additional care must be taken to
also ensure the recovered (unique) solution is the ``physically relevant'' one.
The numerical challenges, on the other hand, stem from the fact that the
discretization scheme should incorporate information about the boundary
conditions at infinity, also called radiation condition; this can be done
directly, by e.g. expanding the solution in a basis which satisfy the radiation
condition \cite{?}, or indirectly by truncating the domain to a finite size and
imposing instead a carefully chosen boundary condition on the artificial
boundaries of the truncated domain \cite{?}.

In the case of volume discretization methods such as finite difference and
finite elements, a particular class of truncation techniques which enjoys great
popularity due to its accuracy and simplicity is the Perfectly Matched Layers
(PMLs) method \cite{?}. Loosely speaking, the method consists of solving for the
analytic extension of the sought solution which decays exponentially at
infinity. The decay rate of the analytic extension then facilitates the
truncation of the domain upon discretization. On the physical (i.e. real)
domain, PMLs can be reinterpreted as surrounding the computational region by a
material layer, not necessarily physical, which absorbs but does not reflect
incoming waves. A key difference between PMLs and \emph{ad hoc} absorbing
material methods stem from the fact that the PML absorbing material is
``perfect'' in the sense that it is reflectionless at the continuous level (i.e.
before discretization).

An alternative to volume discretization methods are boundary integral equation
methods (BIEs), which rely on the use of a Green function to recast the PDE in
terms of integral operators over the interfaces where the medium changes
property. For piece-wise homogeneous media, this reduces the PDE to integral
equations posed at the interface between the different media. Because boundary
integral methods represent the solution as a sum of fundamental solutions which
satisfy the radiation boundary condition, no domain truncation technique is
required when the interfaces are bounded. This is a well-known advantage of
boundary integral methods for scattering problems. There are interesting cases,
however, where such interfaces are better modeled as infinite. Examples include
the scattering of an object near the ground \cite{}, the propagation of elastic
waves under the surface of the earth \cite{}, and wave-guides \cite{}. In such
cases a truncation technique is required, at least in the direction parallel to
the infinite interfaces, so as to reduce the problem to a finite subdomain where
the equations can be discretized.

In order to handle infinite interfaces in a boundary integral equation context,
a few options are available. For relatively simple geometries, one can construct
a problem specific Greens function which incorporates the imposed boundary
condition on all but a bounded portion of the interface, thus reducing the
problem again to integrals over bounded curves/surfaces. This has the advantage
of being conceptually simple provided such problem-specific Greens function can
be efficiently computed. Unfortunately, for all but the simplest geometries, the
representation of the problem specific Greens function involves challenging
integrals which must be approximated numerically \cite{}. Alternatives which
rely instead on the use of the free-space Greens function, readily available for
many PDEs of physical relevance, have also been developed. For example in
\cite{bruno2016windowed} a high-order method called the Windowed-Green function
(WGF) was introduced in order to truncate the integrals stemming from the
boundary integral representation of the unbounded interfaces. More recently,
\cite{lu2018perfectly} proposed a technique based on combining perfectly matched
layers and a boundary integral representation on a certain truncated domain in
the context of Helmholtz scattering problems. Due to the exponential decay of
the Helmholtz Green function on the complex-scaled PML layer, the kernels of the
integral operators over the infinite interfaces become exponentially small,
allowing for an efficient truncation with errors that decay exponentially
respect to a truncation parameter $\ell$ controlling the PML length. It is worth
mentioning the naive approach of simply truncating the domain far enough from
the region of interest is rather unsatisfactory, specially for non-dissipative
PDEs, due to the slow (algebraic) decay of the Greens function; in fact, as we
will argue in this paper, such an abrupt truncation may not even converge in the
zero-frequency regime, wherein the free-space Green function is no longer
decaying.

In this paper we build on the work of~\cite{lu2018perfectly} to develop a
complex-scaled boundary integral equation method for the classical harmonic
water-waves problem. The water-waves problem presents some interesting and novel
challenges, of both theoretical and numerical nature, which are related to the
fact that the free-space Green function is non-oscillatory, and as such its
analytic extension does not become exponentially small inside the PML region.
Despite this undesirable feature of the free-space Green function (which in fact
grows logarithmically at infinity in two-dimensions), we will argue that the
solution can be analytically extended into the complex plane so as to become
exponentially decreasing, justifying at least formally the truncation of certain
integrals over unbounded domains. In more detail, letting $\Omega \subset
\mathbb{R}^2$ denote the (unbounded) fluid domain bounded above by the
linearized free surface $\Gamma_f$ and below by a bottom topography $\Gamma_b$,
and letting $\Gamma_o$ denote the boundary of solid obstacles (either fully or
partially) immersed in the fluid, we seek to solve the following equations:
%
\begin{subequations}
  \label{eq:water-waves-system}
  \begin{align}  
    \label{eq:laplace-equation}
    \Delta \varphi &=0, \quad \bx \in \Omega,\\
    \label{eq:free-surface-bc}
    \nabla \varphi \cdot \bn + \frac{\omega^2}{g}\varphi &=f_1, \quad \bx \in \Gamma_f,\\
    \label{eq:solid-bc}  
    \nabla \varphi \cdot \bn &=f_2, \quad \bx \in \Gamma_s \cup \Gamma_o,\\
    \label{eq:radiation-condition}  
    \lim_{r \to \infty} r^{\frac{d-2}{2}} \left( \partial_{r} \varphi - i k_0 \varphi \right) &= 0,
  \end{align}
\end{subequations}
%
with $k_0 \in \mathbb{R}^+$ in~\cref{eq:radiation-condition} being the solution of
$k\tanh(kd) = \omega^2/g$, and $r = (x_1,\ldots,x_{d-1})$. Throughout this paper $\bx = (x_1,\ldots,x_d) \in \R^d$
denotes a target point, $\bx_\parallel = (x_1,\ldots,x_{d-1})$ are the horizontal
components of the target point, and $\bn$ is the unit normal pointing towards
the fluid domain $\Omega$. Furthermore, the free surface is always located at $x_d = 0$, so that $\Gamma_f =
\R \times \{0\} \setminus \Omega_o$, and the topography is of constant depth $h$
outside a bounded domain (i.e. $\Gamma_b \setminus \R \times \{-h\}$ is
bounded). Finally, for notational simplicity, it will be convenient to define a
global impedance-like coefficient $\alpha$, and global source term $f$ as 
%
\begin{align}
  \label{eq:global-defs}  
  \alpha(\bx) = \begin{cases}
    -\omega^2/g  &\bx \in \Gamma_f\\
    0   &\bx \in \Gamma_b \cup \Gamma_o
  \end{cases}, \quad
  f(\bx) = \begin{cases}
    f_1  &\bx \in \Gamma_f,\\
    f_2  &\bx \in \Gamma_b \cup \Gamma_o,
  \end{cases}, 
\end{align}
% 
so that the boundary conditions~\cref{eq:free-surface-bc,eq:solid-bc} become simply
%
\begin{align}
  \label{eq:global-bc}  
  \nabla \varphi \cdot \bn - \alpha(\bx) \varphi = f, \quad \bx \in \Gamma
\end{align}
%

The remaining of the paper is organized as follows. In
\cref{sec:problem-formulation} we present the problem under consideration, and
introduce the transformed problem under the PML change-of-variables. We focus in
particular on the non-orthogonal PMLs, and show ...

\section{Complex scaled integral equations}

In this section we derive the main result of this paper: a complex scaled
integral equation for solving~\cref{eq:water-waves-system} with the desirable
property that its solutions are exponentially decreasing as $x_1 \to \pm
\infty$. To this end, we begin by recalling some classical results about the
modal decomposition of $\phi$ for a channel of constant (finite) depth. 

\begin{proposition}[Modal decomposition in two-dimensions]
  \label{prop:modal-decomposition}
  Suppose that problem \cref{eq:water-waves-system} is well-posed. Then its
  solution $\varphi$ for $x_1>L$ can be written as
  \begin{align}
    \varphi^+(\bx) = \sum_{n=0}^{\infty} A_n^+ e^{-k_n x_1} \Psi_n(x_2)
  \end{align}
  where $k_0 = ik$, and $k_n$ is the root of $k \tan(kh) + \omega^2/g$ in the
  interval $[(n-1/2)\pi/h, n\pi/h]$, and $\Psi_n(x_2) = c_n \cos(k_n (x_2 +
  h))$, with $c_n = \sqrt{ \frac{1}{2} + \frac{\sin(2 k_n h)}{4k_n h}}$ an $L^2$
  normalization constant. Similarly, the solution for $x_1 < -L$ can be written
  as
  \begin{align}
    \varphi^-(\bx) = \sum_{n=0}^{\infty} A_n^- e^{k_n x_1} \Psi_n(x_2)
  \end{align}
\end{proposition}
\begin{proof}
The proof follows by separation of variables, together with the fact that
$\left\{\Psi_n \right\}_{n=0}^\infty$ forms an orthogonal basis for $L^2([0,-d])$.
See e.g.~\cite[section 2.1]{linton2001handbook}.
\end{proof}

~\Cref{prop:modal-decomposition} gives an explicit decomposition of the
solution, for $|x_1| > L$, in terms of one propagative mode and an infinitude of
evanescent modes. Furthermore, as shown next, it provides a simple way to show
that $\phi$ admits an analytic extension to complex values of $x_1$ (for any $d
\leq x_2 \leq 0$):

\begin{proposition}[Analytic extension]
  \label{prop:analyticity}
  Suppose that problem \cref{eq:water-waves-system} is well-posed. Then its solution $\varphi$ for $x_1>L$ (resp. $x_1<-L$)  has an analytic extension to complex values of $x_1$ in $\Re(x_1)>L$ (resp. $\Re(x_1)<-L$). 
\end{proposition}
\begin{proof}
Let us consider for instance the side $x_1>L$. It is clear that for any $N \geq
1$, the finite sum 
\begin{align}
  \varphi_N^+(\bx) = \sum_{n=0}^{N} A_n^+ e^{-k_n x_1} \Psi_n(x_2)
\end{align}
is well-defined for complex values of $x_1$, and is a holomorphic function of
$x_1$. Moreover, it converges to $\varphi(x_1,x_2)$ when $N\to +\infty$,
uniformly in all compact subsets of $\Re(x_1)>L$.  Finally, thanks to Morera
theorem, a uniform limit of a sequence of holomorphic functions is itself
holomorphic.
\end{proof}

We consider now the PML change of variables, whose main goal is to transform
(radiative) solutions of~\cref{eq:water-waves-system} into exponentially
decaying functions. We will present the method in a somewhat general context so
as to reuse the same notation when discussing the three-dimensional problem. 

Let $\btau : \R^d \to \C^d$ be a vector-valued function representing a complex
change of variables mapping points $\bx$ on our physical domain to complex points $\tilde{\bx} = \btau(\bx)$. This change of variables transforms $\Omega$ into $\tilde{\Omega}$, defined as
\begin{align}
  \tilde{\Omega} = \left\{ \tilde{\bx} = \btau(\bx) : \bx \in \Omega \right\}
\end{align}
The boundaries $\Gamma_f$ and $\Gamma_s$ are transformed into $\tilde{\Gamma}_f$
and $\tilde{\Gamma}_s$ similarly. Assuming that $\phi$ admits an analytic
extension into $\tilde{\Omega}$, it follows that its analytic extension
satisfies the following complexified PDE on $\tilde{\Omega}$:
%
\begin{subequations}
\label{eq:complex-equations}  
\begin{align}  
    \label{eq:pml-laplace-equation}
    \Delta_{\tilde{\bx}} \varphi &=0 \quad(\tilde \Omega)\\
    \label{eq:pml-bc1} 
    \nabla_{\tilde{\bx}} \varphi \cdot \bn - \alpha \varphi &= f \quad   (\tilde{\Gamma})
\end{align}
\end{subequations}
%
where $\Delta_{\tilde{\bx}}$ denotes the Laplacian respect to the $\tilde{\bx}$
variables; i.e. $\Delta_{\tilde{\bx}} = \sum_{n=1}^d \partial^2 / \partial
\tilde{x}_n^2$. (Similarly $\nabla_{\tilde{\bx}}$ denotes the gradient respect
to $\tilde{\bx}$.)

Letting $\tvarphi(\bx) = \varphi(\tilde{\bx}) = (\varphi \circ \btau)(\bx)$, we
may transform \cref{eq:complex-equations} back to the real variables $\bx$ by
means of $\bx =\tau^{-1}(\tilde{\bx})$ (which is assumed to be invertible).
Denoting by $J$ the Jacobian matrix of the transformation $\tilde{\bx} =
\tau(\bx)$, given by $J_{ij}=\pder{\tau_i}{x_j} = \tau_{i,j}$, application of
the chain rule (i.e. $\nabla_{\tilde{\bx}} \to (J^{-1})^t \nabla_{\bx}$)
together with algebraic manipulations yields the following PML-transformed water
waves problem:
\begin{subequations}
\label{eq:pml-water-waves}  
\begin{align}
    \label{eq:pml-laplace-real}
    \nabla \cdot \left( A \nabla \tvarphi(\bx) \right) &= 0, \quad (\Omega)\\
    \label{eq:pml-bc-real} 
    \nabla \tvarphi \cdot J^{-1}\bn - \alpha \tvarphi &= f \quad (\Gamma)
\end{align}
\end{subequations}
where $A = |J|J^{-1}(J^{-1})^t$, with $J^{-1}$ denoting the inverse of $J$ and
$|J|$ its determinant. System~\cref{eq:pml-water-waves} is the PML-transformed
system of equations that we next seek to solve by a boundary integral equation
method.

We suppose throughout this paper that $\tau_d(\bx) = x_d$, so that the
$d$-direction (corresponding to the \emph{depth} direction) is not affected by
the complex-stretching $\tau$. This is a natural assumption since the depth
direction, even if infinite, supports no waves. It then follows that $(J^{-1})^t
\bn = \bn$, and therefore $|J| J^{-1} \bn = A \bn$; this identity can be used to
recast the boundary condition~\cref{eq:pml-bc-real} using the co-normal
derivative of $\tvarphi$:
\begin{align}
  \label{eq:pml-bc-real-2}
  \nabla \tvarphi \cdot A\bn - |J| \alpha \tvarphi &= f \quad (\Gamma)
\end{align}

In what follows, we apply the general theory of boundary integral equations for
strongly elliptic systems~\cite{mclean2000strongly}. This imposes restrictions
on $\btau$; in particular, system \cref{eq:complex-equations} must remain
strongly elliptic, as per the following definition~\cite[equation
4.7]{mclean2000strongly}:
\begin{definition}[Strongly elliptic PDE]
  The scalar partial differential equation
  \begin{align}
    \nabla \cdot \left( A(\bx) \nabla \tvarphi(\bx) \right) = 0,
  \end{align}
  where $A$ is a symmetric $d \times d$ matrix over $\mathbb{C}$ and
  $\tvarphi : \R^d \to \C$, is strongly-elliptic on the domain $\Omega$
  if $\exists \delta > 0$ s.t.
  \begin{align}
    \mathrm{Re}\left( \xi^* A(\bx) \xi \right) > \delta |\xi|^2,
  \end{align}
  for all $\bx \in \Omega$, $\xi \in \C^{d}$, where $\xi^*$ denotes the complex
  transpose (adjoint) of $\xi$.
\end{definition}

The following proposition provides necessary and sufficient conditions under
which \cref{eq:pml-laplace-real} is strongly elliptic:
%
\begin{proposition}
  \label{pr:algebraic-condition-strongly-elliptic}
  The system \cref{eq:pml-laplace-real} is strongly elliptic if and only if
  $\mathrm{Re}(A)$ is positive definite.
\end{proposition}
\begin{proof}
  Since $A$ was assumed symmetric, it follows that
  \begin{align}
  2 \mathrm{Re}(\xi^* A \xi) = \xi^* A \xi + \overline{\xi^* A \xi}= \xi^* A \xi + \xi^* \bar{A} \xi = 2 \xi^* \mathrm{Re}(A) \xi
  \end{align}
  Splitting $\xi = \xi_r + i\xi_i$ into its real and imaginary parts, with
  $\xi_r,\xi_i \in \R^d$, we have that
  \begin{align}
  \xi^* \mathrm{Re}(A) \xi = \xi_r^t \mathrm{Re}(A) \xi_r + \xi_i^t \mathrm{Re}(A) \xi_i
  \end{align}
  If $\mathrm{Re}(A)$ is positive definite, we have that $\exists \delta > 0$ s.t.
  \begin{align}
  \xi^* \mathrm{Re}(A) \xi > \delta (\xi_r^2 + \xi_i^2) = \delta |\xi|^2
  \end{align}
  which proves the result one way. If $\mathrm{Re}(A)$ is not positive definite,
  then $\exists v \in \R^2$ s.t. $v^t \mathrm{Re}(A) v \leq 0$. Letting $\xi = v$
  the other direction is proved.
\end{proof}

\Cref{pr:algebraic-condition-strongly-elliptic} provides an algebraic criterion,
dependent only on the chosen change of variable $\btau$, to determine whether
\cref{eq:pml-laplace-real} is strongly elliptic on $\Omega$. In most of what
follows we will employ the so-called orthogonal PMLs, which yield a
\emph{diagonal} Jacobian matrix $J$ (and thus a diagonal $A$). Under this
simplifying assumption, directly verifying that the PML-tranformed PDE remains
strongly elliptic is relatively straightforward. The more general case of
non-orthogonal PMLs is a more involved, as it imposes constraints on the
derivatives of $\btau$ respect to the vertical variable $x_d$. 

Under the assumption $\btau$ is such that~\cref{eq:pml-laplace-real} is strongly
elliptic in the sense of~\cref{pr:algebraic-condition-strongly-elliptic}, it
follows from the general theory of boundary integral
equations~\cite{mclean2000strongly} that $\tvarphi$ admits Green-like integral
representation formulae. Indeed, letting $\tilde{G}$ be the free-space Green's
function associated with \cref{eq:pml-laplace-real}, and $\Omega_M = \left\{\bx
\in \Omega : |\bx_{\perp}| <  M \right\}$ be a truncated domain with $M>L$, the
following integral representation formula holds:
\begin{align}
  \label{eq:greens-representation}
  \tvarphi(\br) = \mathcal{D}[\gamma_0\tvarphi](\br) - \mathcal{S}[\tilde{\gamma}_1 {\tvarphi}](\br) \quad \mbox{for} \quad \br \in \Omega_M,
\end{align}
where $\gamma_0$ and $\gamma_1$ are the generalized Dirichlet and Neumann trace
operators, formally defined as 
\begin{subequations}
\label{eq:traces}  
\begin{align}
  \label{eq:dir-trace}
  \gamma_0[\sigma](\bx) &= \lim_{\epsilon \to 0^+} \sigma(\bx + \epsilon \bn(\bx)),\\
  \label{eq:neu-trace}
  \gamma_1[\sigma](\bx) &= \lim_{\epsilon \to 0^+} \nabla \sigma(\bx + \epsilon \bn(\bx)) \cdot A^t \bn(\bx),
\end{align}
\end{subequations}
and $\mathcal{S}$ and $\mathcal{D}$ denote the single- and double-layer
potentials, given by:
\begin{subequations}\label{eq:potentials}
  \begin{align}
    \label{eq:SL-potential}  
    \mathcal{S}[\sigma](\br) &:= \int_{\partial \Omega_M} {\tilde{G}}(\br, \by)\sigma(\by) \de s(\by), \\
    \label{eq:DL-potential}  
    \mathcal{D}[\sigma](\br) &:= \int_{\partial \Omega_M} \tilde{\gamma}_{1,\by}{\tilde{G}}(\br, \by) \sigma(\by) \de s(\by).
  \end{align}
\end{subequations}
(The notation $\tilde{\gamma}_{1,\by}$ in~\eqref{eq:DL-potential} means the
derivatives in the operator $\tilde{\gamma}_1$ are with respect to the $\by$
variable). 

Using the fact that $\tvarphi$ decays exponentially with $M$ on the lateral
boundaries $$\Theta = \left\{ (\bx_\perp, d) : |\bx_\perp| = M, d \in [0,-h]
\right\} \subset \partial\Omega_M,$$ while $\tilde{G}$ grows at most
logarithmically, we may take the limit $M \to \infty$ and replace $\Omega_M$
(resp. $\partial \Omega_M$) by $\Omega$ (resp. $\partial \Omega = \Gamma$)
in~\cref{eq:greens-representation} (resp.~\cref{eq:potentials}). This justifies
the validity of the complex-scaled boundary integral representation on the
domain $\Omega$ containing and unbounded interface $\Gamma$. 

Finally, applying the trace operators $\gamma_0$ and $\gamma_1$
to~\cref{eq:greens-representation}, and accounting for the jump in the
double-layer potential across $\Gamma$, we arrive at the well-known Green's
identity:
\begin{align}
  \label{eq:greens-formula}
  \frac{\gamma_0 \tvarphi(\bx)}{2} &=  D[\gamma_0 \tvarphi](\bx) - S[{\gamma}_1 \tvarphi](\bx) \quad \mbox{for} \quad \bx \in \Gamma,
\end{align}
where $S$ and $D$ are the single- and double-layer operators,
defined as:
\begin{subequations}
  \label{eq:calderon-operators}
  \begin{align}
  \label{eq:SL-operator}  
  S[\varphi](\bx) &:= \int_\Gamma \tilde{G}(\bx,\by) \varphi(\by) \de s(\by)\\
  \label{eq:DL-operator}  
  D[\varphi](\bx) &:= {\rm p.v.}\! \int_\Gamma\left(\tilde{\gamma}_{1,\by}\tilde{G}(\bx,\by)\right)
  \varphi(\by) \de s(\by)
\end{align}\label{eq:BIOS}\end{subequations}
where p.v. in front of~\cref{eq:DL-operator} means the integral is to be
interpreted as a Cauchy principal-value.

Interestingly, owing to the fact~\cref{eq:pml-laplace-real} was obtained
through a change of variables of a constant coefficient PDE with known
fundamental solution, the free-space greens function $\tilde{G}(\bx,\by)$ is explicitly given by $\tilde{G}(\bx,\by) = G(\tau(\bx),\tau(\by))$:
\begin{proposition}[PML Greens function]
  The free-space Greens function associated with~\cref{eq:pml-laplace-real} is
  given by
  \begin{align}
  \tilde{G}(\bx,\by) = G_{\Delta}(\tau(\bx),\tau(\by)) = 
    \begin{cases}
        \frac{-1}{2\pi}\log(\rho(\btau(\bx),\btau(\by))),\quad &d=2 \\
        \frac{1}{4\pi\rho(\btau(\bx),\btau(\by))}, \quad &d=3
    \end{cases},
  \end{align}
  where $\rho(\bx,\by) = \sqrt{(\bx-\by)\cdot(\bx-\by)}$, and where the square
  root is chosen with a branch-cut on the negative real line.
\end{proposition}
\begin{proof}
  For the sake of completeness, we present a formal proof
  in~\cref{sec:greens-function} which makes use only of elementary calculus; the
  reader is directed to \cite{kim2010analysis} for a rigorous treatment. 
\end{proof}

With the validity of the representation formula~\cref{eq:greens-representation}
established, and the Green's function readily available, we can now
reformulate~\cref{eq:pml-laplace-real} using boundary integral equations.
Using~\cref{eq:pml-bc-real-2} to replace $\gamma_1\tvarphi := \nabla \tvarphi
\cdot A \bn$ in
favor of $\gamma_0 \tvarphi$ in~\cref{eq:greens-formula}, we obtain the
following second-kind boundary integral equation (BIE) for the unknown velocity potential
$\tvarphi$:
\begin{align}
  \label{eq:BIE}
  \frac{\tvarphi(\bx)}{2} - D[\tvarphi](\bx) + S\left[\alpha\left|J\right|\tvarphi\right](\bx) &= S[f](\bx), \quad \bx \in \Gamma.
\end{align}

Recalling the definition of $\alpha$ from~\cref{eq:global-defs}, we may simplify
equation~\cref{eq:BIE} to obtain:
\begin{align}
  \label{eq:BIE-2}
  \frac{\tvarphi(\bx)}{2} - D[\tvarphi](\bx) + \frac{\omega^2}{g}S_{\Gamma_f}\left[\left|J\right|\tvarphi\right](\bx) &= S[f](\bx), \quad \bx \in \Gamma,
\end{align}
where the $\Gamma_f$ subscript in the single-layer operator means that
integration is performed only over the free-surface. Interestingly, the proposed
complex-scaled formulation~\cref{eq:BIE-2} depends linearly on the frequency
$\omega$; this contrasts e.g. formulations based on the water-waves Green's
function, for which the $\omega$ dependency appears (nonlinearly) inside the
kernels of the integral operators. We will explore this feature to compute
resonant frequencies in~\cref{sec:?} through a linear (generalized) eigenvalue
problem.

% Writing down the explicit expression for the double-layer, we get
% \begin{align}
%   D[\tvarphi] = \int_\Gamma \nabla_\by G(\tau(\bx),\tau(\by)) \cdot A(\by) \bn(\by) \tvarphi \de s_\by
%   = \int_\Gamma (\nabla_\by G)(\tau(\bx),\tau(\by)) \cdot \bn(\by) |J(\by)| \tvarphi \de s_\by.
% \end{align}
% Defining $\sigma = |J| \tvarphi$, we get 

In the following section, we will seek to solve \cref{eq:BIE} by first
truncating the unbounded curves at some finite distance, and then by
discretizing the (singular) integral operators by an appropriate quadrature
rule. 

\begin{remark}[Irregular frequencies]
  It remains unclear whether~\cref{eq:BIE} is uniquely solvable for all
  frequencies $\omega$. If there exists values of $\omega$ such that the
  homogenous system ($f=0$) admits a non-trivial solution, then there is no
  uniqueness of the BIE. The unboundedness of $\Gamma$, together with the fact
  that $|J|$ is not constant, however, makes it difficult to address this
  question.
\end{remark}

\begin{remark}[Indirect formulation]
  Another approach to derive a boundary integral equation
  for~\cref{eq:water-waves-system} is to use an \emph{indirect} approach, where
  we seek the velocity potential in the form $\tilde{\phi}(\br) =
  \mathcal{S}[\sigma](\br)$, where $\sigma$ is an unknown density to be found by
  imposing the boundary conditions. This leads to an equation similar
  to~\cref{eq:BIE}; we have chosen to pursue a \emph{direct} formulation in
  order to exploit the fact that $\tvarphi$ is exponentially decreasing as
  $|\bx_\perp| \to \infty$. For the indirect formulation, that is likely to be
  true for the density $\sigma$, but we have not explored that option in great
  detail. 
\end{remark}

\section{Discretization}

\section{Numerical results}

\subsection{Convergence with respect to PML}

We now show 

% \appendix

% \section{Ellipticity condition}

% For the two-dimensional problem, we have that 
% \begin{align}
%   J = \begin{bmatrix}
%     \tau_{1,1} & \tau_{1,2}\\
%     0 & 1
%   \end{bmatrix}
%   ,\quad 
%   J^{-1} = \begin{bmatrix}
%     1/\tau_{1,1} & -\tau_{1,2}/\tau_{1,1}\\
%     0 & 1
%   \end{bmatrix}
% \end{align}
% and so
% \begin{align}
%   A = |J|J^{-1}(J^{-1})^t = 
%   \frac{1}{\tau_{1,1}}
%   \begin{bmatrix}
%     1 + \tau_{1,2}^2 & -\tau_{1,2}\tau_{1,1} \\
%     -\tau_{1,2}\tau_{1,1} & \tau_{1,1}^2
%   \end{bmatrix}
% \end{align}
% Assuming $\tau_{1,2}$ is purely imaginary, we have that
% \begin{align}
%   \mathrm{Re}(A) = 
%   \begin{bmatrix}
%     \mathrm{Re}\left(\frac{1 + \tau_{1,2}^2}{\tau_{1,1}}\right) & 0 \\
%     0 & \mathrm{Re}(\tau_{1,1})
%   \end{bmatrix}
% \end{align}
% Considering a change of variables of the form $\tau_i = x_i + i\sigma_i(\bx)$,
% we arrive at the following simple condition:
% \begin{align}
%   |\sigma_{1,2}| < 1
% \end{align}


% While in \cref{sec:orthogonal-pml} we considered the special case where
% $\tau(\bx) = (\tau_1(x_1),\tau_2(x_2))$, we present now the generic case with
% $\tau(\bx) = (\tau_1(\bx),\tau_2(\bx))$. The matrix $A$ is then given by
% \begin{align}
%   A = \frac{1}{|J|}
%   \begin{bmatrix}
%     \tau_{2,2}^2 + \tau_{1,2}^2 & -\tau_{2,1}\tau_{2,2} - \tau_{1,2}\tau_{1,1} \\
%     -\tau_{2,1}\tau_{2,2} - \tau_{1,2}\tau_{1,1}  & \tau_{1,1}^2 + \tau_{2,1}^2
%   \end{bmatrix},
% \end{align}

% Determining ellipticity of this matrix (i.e. finding conditions on $\tau$ under
% which $\mathrm{Re}(A)$ is positive definite) in the general case is a somewhat
% cumbersome calculation which we deferred to \cref{sec:ellipticity-of-A}. For the
% numerical examples we consider in \cref{sec:numerical-examples}, it suffices to
% focus on the uni-axial (but non-orthogonal) case, where we assume $\tau_2(\bx) =
% x_2$. The matrix $A$ then simplifies to
% \begin{align}
%   A  =   \begin{bmatrix}
%     \frac{1 - \sigma_{1,2}^2}{1 + i\sigma_{1,1}} & -i\sigma_{1,2} \\
%     -i\sigma_{1,2} & 1 + i\sigma_{1,1}
%   \end{bmatrix}, \quad
%                      \mathrm{Re}(A) =   \begin{bmatrix}
%     \frac{1 - \sigma_{1,2}^2}{1+\sigma_{1,1}^2} & 0 \\
%     0 & 1
%   \end{bmatrix},
% \end{align}
% %
% and strong ellipticity is equivalent to the simple condition:
% \begin{align}
%   \label{eq:ellipticity-condition-nonorthogonal}
%   \left| \sigma_{1,2}(\bx) \right| < 1 \quad \mbox{for all} \quad \bx \in \Omega.
% \end{align}

% In order to give a geometric interpretation to this condition, it is useful to
% consider a simple case given by
% \begin{align}
%   \tau_1 =
%   \begin{cases}
%     x_1 + i \beta (x_1 + \frac{1}{\alpha} x_2 + a) \quad &\mbox{for} \quad x_1 < -\frac{1}{\alpha} x_2 - a, \\
%     x_1 \quad &\mbox{for} \quad |x_1| < \frac{1}{\alpha} x_2 + a, \\
%     x_1 + i \beta (x_1 - \frac{1}{\alpha} x_2 - a) \quad &\mbox{for} \quad x_1 > \frac{1}{\alpha} x_2 + a.
%   \end{cases}
% \end{align}
% The parameters $\alpha$ and $a$ determine the region where the PML is applied,
% and the parameter $\beta$ controls the attenuation strength inside the PML layer
% (see \cref{fig:sch-pml}). By taking taking the limit $\alpha \to \infty$, for
% instance, we recover the case of an orthogonal PML discussed in
% \cref{sec:orthogonal-pml}.

% We are interested in the analytic extension of $\varphi$ when it is
% exponentially decaying at infinity. This leads to consider the analytic
% extension of $\varphi$ for $\Re(x_1)>L$  (resp. $\Re(x_1)<-L$) only for
% $\Im(x_1)>0$ (resp. $\Im(x_1)<0$), so that the propagating surface mode becomes
% exponentially decaying at infinity. Letting $\btau : \R^2 \to
% \C^2$ mapping points $\bx$ on the physical domain to $\tilde{\bx} = \btau(\bx)$.
% Then, assuming that $\tau(\bx)$ is the identity map for $|x_1|<L$, and defining
% $\tilde{\varphi}(x_1,x_2) = \varphi(\tau(x_1),x_2)$ as the analytic 

\bibliographystyle{abbrv}
\bibliography{references}

\end{document}
